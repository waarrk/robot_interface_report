\documentclass[dvipdfmx,titlepage,a4j]{jsarticle}

\usepackage{url}
\usepackage{graphicx}
\usepackage{listings,jvlisting}
\usepackage{amsmath,amssymb}
\usepackage{graphicx}
\usepackage[yen]{okuverb}
\usepackage{ascmac}
\usepackage{fancybox}
\usepackage{fancyvrb}
\usepackage{fancyhdr}
\usepackage[table]{xcolor}
\usepackage{lastpage}
\usepackage{caption}
\usepackage{subcaption}
\usepackage{here}
\usepackage{textcomp}

\fancypagestyle{foot}
{
\fancyhead[C]{}
\fancyfoot[C]{\thepage / \pageref{LastPage}}
\renewcommand\headrulewidth{0.4pt}
}

\definecolor{lightblue}{RGB}{220,230,255}
\definecolor{lightpink}{RGB}{255,230,210}

%ここからソースコードの表示に関する設定
\lstset{
  language={C++},
  basicstyle={\ttfamily},
  identifierstyle={\small},
  commentstyle={\smallitshape},
  keywordstyle={\small\bfseries},
  ndkeywordstyle={\small},
  stringstyle={\small\ttfamily},
  frame={tb},
  tabsize={2},
  breaklines=true,
  columns=[l]{fullflexible},
  numbers=left,
  xrightmargin=0zw,
  xleftmargin=3zw,
  numberstyle={\scriptsize},
  stepnumber=1,
  numbersep=1zw,
  lineskip=-0.5ex
}

\title{タイトル}
\author{waarrk}
\date{2023年2月1日}

\begin{document}

\begin{titlepage}
    \centering
    \vspace*{2cm}

    \vspace{1cm}

    {\LARGE \textbf{ロボットインターフェース設計論 課題}}

    \vspace{0.5cm}

    {2024年度開講 ロボットインターフェース設計論レポート}

    \vspace{1.5cm}

    {\textbf{千葉工業大学 先進工学部 未来ロボティクス学科}\\}
    {\textbf{22C1704 鷲尾 優作}}

    \vfill

    {\large 2024年12月10日}

    \vspace{1cm}
\end{titlepage}

\paragraph{ 問題 1\\}
素数 7 で割った余り[0,1,2,3,4,5,6]を元とし、次の演算が定義されているガロア体 GF(7)について、(1)~(6)の式の値を求めよ。

\begin{minipage}{0.45\hsize} \centering
\[
\begin{array}{c|ccccccc}
+ & 0 & 1 & 2 & 3 & 4 & 5 & 6 \\ \hline
0 & 0 & 1 & 2 & 3 & 4 & 5 & 6 \\
1 & 1 & 2 & 3 & 4 & 5 & 6 & 0 \\
2 & 2 & 3 & 4 & 5 & 6 & 0 & 1 \\
3 & 3 & 4 & 5 & 6 & 0 & 1 & 2 \\
4 & 4 & 5 & 6 & 0 & 1 & 2 & 3 \\
5 & 5 & 6 & 0 & 1 & 2 & 3 & 4 \\
6 & 6 & 0 & 1 & 2 & 3 & 4 & 5 \\
\end{array}
\]
\end{minipage} \begin{minipage}{0.45\hsize} \centering
\[
\begin{array}{c|ccccccc}
\times & 0 & 1 & 2 & 3 & 4 & 5 & 6 \\ \hline
0 & 0 & 0 & 0 & 0 & 0 & 0 & 0 \\
1 & 0 & 1 & 2 & 3 & 4 & 5 & 6 \\
2 & 0 & 2 & 4 & 6 & 1 & 3 & 5 \\
3 & 0 & 3 & 6 & 2 & 5 & 1 & 4 \\
4 & 0 & 4 & 1 & 5 & 2 & 6 & 3 \\
5 & 0 & 5 & 3 & 1 & 6 & 4 & 2 \\
6 & 0 & 6 & 5 & 4 & 3 & 2 & 1 \\
\end{array}
\]
\end{minipage}
\\

求めるべき式は以下である。

\[
(1)\;2 + 4, \quad 
(2)\;5 - 2, \quad 
(3)\;2 - 5, \quad 
(4)\;4 \times 3, \quad 
(4)\;4 \div 3, \quad 
(6)\;1 \div 6.
\]

\paragraph{ 回答 1\\}
7で割った余りを元とする演算であるから、以下のように計算できる。

\[
\begin{aligned}
(1)\;2 + 4 &\equiv 6 \pmod{7}\\[6pt]
(2)\;5 - 2 &= 5 + (-2) = 5 + 5 \equiv 10 \equiv 3 \pmod{7}\\[6pt]
(3)\;2 - 5 &= 2 + (-5) = 2 + 2 \equiv 4 \pmod{7}\\[6pt]
(4)\;4 \times 3 &\equiv 12 \equiv 5 \pmod{7}\\[6pt]
(5)\;4 \div 3 &= 4 \times 3^{-1} = 4 \times 5 \equiv 20 \equiv 6 \pmod{7}\\[6pt]
(6)\;1 \div 6 &= 1 \times 6^{-1} = 1 \times 6 \equiv 6 \pmod{7}
\end{aligned}
\]


\paragraph{ 問題 2\\}
下図のように7文字からなる文字列の各文字の7ビットの
ASCIIコード(表1)に偶数パリティで計算された垂直水平パリティが付与された64ビットの
符号語が送信され,式(1)の符号語を受信した.
1ビットまでの誤りがあることを考慮して送信された7文字の正しい“文字列”を復号せよ.

\begin{center}
    \small
    \setlength{\tabcolsep}{12pt}
    \renewcommand{\arraystretch}{1.2}

\begin{tabular}{
    c
    | c c c c c c c 
    >{\columncolor{lightpink}}c
}
 & b$_1$ & b$_2$ & b$_3$ & b$_4$ & b$_5$ & b$_6$ & b$_7$ & b$_8$ \\ \hline
W$_1$ & x$_{11}$ & x$_{12}$ & x$_{13}$ & x$_{14}$ & x$_{15}$ & x$_{16}$ & x$_{17}$ & x$_{18}$ \\
W$_2$ & x$_{21}$ & x$_{22}$ & x$_{23}$ & x$_{24}$ & x$_{25}$ & x$_{26}$ & x$_{27}$ & x$_{28}$ \\
W$_3$ & x$_{31}$ & x$_{32}$ & x$_{33}$ & x$_{34}$ & x$_{35}$ & x$_{36}$ & x$_{37}$ & x$_{38}$ \\
W$_4$ & x$_{41}$ & x$_{42}$ & x$_{43}$ & x$_{44}$ & x$_{45}$ & x$_{46}$ & x$_{47}$ & x$_{48}$ \\
W$_5$ & x$_{51}$ & x$_{52}$ & x$_{53}$ & x$_{54}$ & x$_{55}$ & x$_{56}$ & x$_{57}$ & x$_{58}$ \\
W$_6$ & x$_{61}$ & x$_{62}$ & x$_{63}$ & x$_{64}$ & x$_{65}$ & x$_{66}$ & x$_{67}$ & x$_{68}$ \\
W$_7$ & x$_{71}$ & x$_{72}$ & x$_{73}$ & x$_{74}$ & x$_{75}$ & x$_{76}$ & x$_{77}$ & x$_{78}$ \\
\rowcolor{lightblue} 
W$_8$ & x$_{81}$ & x$_{82}$ & x$_{83}$ & x$_{84}$ & x$_{85}$ & x$_{86}$ & x$_{87}$ & x$_{88}$ \\
\end{tabular}

\end{center}
\vspace{1em}

\[
\text{受信語 }(W_1 W_2 W_3 W_4 W_5 W_6 W_7 W_8) =
\begin{aligned}
\\
(x_{11}x_{12}x_{13}x_{14}x_{15}x_{16}x_{17}x_{18}\,
x_{21}x_{22}x_{23}x_{24}x_{25}x_{26}x_{27}x_{28}\,\\
x_{31}x_{32}x_{33}x_{34}x_{35}x_{36}x_{37}x_{38}\,
x_{41}x_{42}x_{43}x_{44}x_{45}x_{46}x_{47}x_{48}\, \\
x_{51}x_{52}x_{53}x_{54}x_{55}x_{56}x_{57}x_{58}\,
x_{61}x_{62}x_{63}x_{64}x_{65}x_{66}x_{67}x_{68}\,\\
x_{71}x_{72}x_{73}x_{74}x_{75}x_{76}x_{77}x_{78}\,
x_{81}x_{82}x_{83}x_{84}x_{85}x_{86}x_{87}x_{88})\\
\end{aligned}
\]
\[
= 10101100\,11000011\,11110100\,11010010\,11011110\,11101011\,11100111\,10001010 \cdots (1)
\]

\begin{table}[h]
    \centering
    \caption{ASCIIコード}
    \begin{tabular}{|c|c|c|c|c|c|c|c|c|c|c|c|c|c|c|c|c|}
        \hline
         & \_0 & \_1 & \_2 & \_3 & \_4 & \_5 & \_6 & \_7 & \_8 & \_9 & \_A & \_B & \_C & \_D & \_E & \_F \\ \hline
        2\_ &   & ! & " & \# & \$ & \% & \& & ' & ( & ) & * & + & , & - & . & / \\ \hline
        3\_ & 0 & 1 & 2 & 3 & 4 & 5 & 6 & 7 & 8 & 9 & : & ; & \textless & = & \textgreater & ? \\ \hline
        4\_ & @ & A & B & C & D & E & F & G & H & I & J & K & L & M & N & O \\ \hline
        5\_ & P & Q & R & S & T & U & V & W & X & Y & Z & \textbackslash & \^{} & \_ & \{ & \} \\ \hline
        6\_ & ` & a & b & c & d & e & f & g & h & i & j & k & l & m & n & o \\ \hline
        7\_ & p & q & r & s & t & u & v & w & x & y & z & \{ & \textbar & \} & \textasciitilde & DEL \\ \hline
    \end{tabular}

    
    \vspace{1em}
    {\small ※ たとえば ‘A’のASCIIコードは,上記表から16進数で0x41,2進数(100 0001)}
    
    \end{table}

\paragraph{ 回答 2\\}
訂正後の7行7ビット(W1~W7, b1~b7)のビット列を示す。
\[
\begin{aligned}
&\quad W_1: 1\,0\,1\,0\,1\,1\,0 \quad(2)\\
&\quad W_2: 1\,1\,0\,0\,0\,0\,1 \quad(2)\\
&\quad W_3: 1\,1\,1\,0\,0\,1\,0 \quad(2)\\
&\quad W_4: 1\,1\,0\,1\,0\,0\,1 \quad(2)\\
&\quad W_5: 1\,1\,0\,1\,1\,1\,1 \quad(2)\\
&\quad W_6: 1\,1\,1\,0\,1\,0\,1 \quad(2)\\
&\quad W_7: 1\,1\,1\,0\,0\,1\,1 \quad(2)\\[6pt]
\end{aligned}
\]
したがって、割り当てられるASCIIコードは以下の通りである。
\[
\begin{aligned}
&\quad W_1: 1010110_2 = 0x56_{16} = \rightarrow\text{'V'}\\
&\quad W_2: 1100001_2 = 0x61_{16} = \rightarrow\text{'a'}\\
&\quad W_3: 1110010_2 = 0x72_{16} = \rightarrow\text{'r'}\\
&\quad W_4: 1101001_2 = 0x69_{16} = \rightarrow\text{'i'}\\
&\quad W_5: 1101111_2 = 0x6F_{16} = \rightarrow\text{'o'}\\
&\quad W_6: 1110101_2 = 0x75_{16} = \rightarrow\text{'u'}\\
&\quad W_7: 1110011_2 = 0x73_{16} = \rightarrow\text{'s'}\\[6pt]
&\text{よって、誤り訂正後の文字列は:}\\
&\boxed{\text{Various}}
\end{aligned}
\]

\paragraph{ 問題 3\\}
下記の生成行列$G$・検査行列$H$が与えられた時に、情報語長4ビットの情報(1 1 1 0)を7ビット長のハミング符号で符号化した符号語を求めよ。

\[
H = \begin{bmatrix}
0 & 1 & 1 & 1 & 1 & 0 & 0 \\
1 & 0 & 1 & 1 & 0 & 1 & 0 \\
1 & 1 & 0 & 1 & 0 & 0 & 1
\end{bmatrix}
\qquad
G = \begin{bmatrix}
1 & 0 & 0 & 0 & 0 & 1 & 1 \\
0 & 1 & 0 & 0 & 1 & 0 & 1 \\
0 & 0 & 1 & 0 & 1 & 1 & 0 \\
0 & 0 & 0 & 1 & 1 & 1 & 1
\end{bmatrix}
\]

※情報$i=(x1,x2,x3,x4)$が与えられた時のハミング符号$w=(x1,x2,x3,x4,c1,c2,c3)$の求め方(※足し算は排他的論理和を適用)

$w=iG$

ハミング符号$w$が与えられた時の、検査の仕方は、$e=Hw^T$を計算し、$e = 0$の時は、誤りなし、$e \neq 0$の時は、$e$と一致する$H$の列の列番号が
誤っているビットの位置を示す。

\paragraph{ 回答 3\\}
\[
\begin{aligned}
&\textbf{情報語:} i = (1,\,1,\,1,\,0) \\[6pt]
&\text{まず } i \text{ と } G \text{ の積を2進数排他的論理和として求める}\\[6pt]
&\text{行列 }G\text{ の各行を }g_1,g_2,g_3,g_4\text{ とすると:}\\
&g_1 = [1,\,0,\,0,\,0,\,0,\,1,\,1]\\
&g_2 = [0,\,1,\,0,\,0,\,1,\,0,\,1]\\
&g_3 = [0,\,0,\,1,\,0,\,1,\,1,\,0]\\
&g_4 = [0,\,0,\,0,\,1,\,1,\,1,\,1]\\[6pt]
&\text{情報語 } i=(x_1,x_2,x_3,x_4)=(1,1,1,0) \text{ より、}\\
&w = x_1g_1 \oplus x_2g_2 \oplus x_3g_3 \oplus x_4g_4 = g_1 \oplus g_2 \oplus g_3\\[6pt]
&g_1 \oplus g_2 = [1,\,0,\,0,\,0,\,0,\,1,\,1] \oplus [0,\,1,\,0,\,0,\,1,\,0,\,1]\\
&\quad = [1\oplus0,\ 0\oplus1,\ 0\oplus0,\ 0\oplus0,\ 0\oplus1,\ 1\oplus0,\ 1\oplus1]\\
&\quad =  (g_1 \oplus g_2) \oplus g_3 = [1,\,1,\,0,\,0,\,1,\,1,\,0] \oplus [0,\,0,\,1,\,0,\,1,\,1,\,0]\\
&\quad = [1\oplus0,\ 1\oplus0,\ 0\oplus1,\ 0\oplus0,\ 1\oplus1,\ 1\oplus1,\ 0\oplus0]\\
&\quad = [1,\,1,\,1,\,0,\,0,\,0,\,0]\\[6pt]
&\text{よって求める符号語は } w = (1,\,1,\,1,\,0,\,0,\,0,\,0) \text{ となる}\\[6pt]
&\boxed{w = (1,1,1,0,0,0,0)}
\end{aligned}
\]

\paragraph{ 問題 4\\}
生成行列 G ・検査行列 H に問題 3 のものを使い、”あ”~”た”に、0000~1111 までを割り当 て、8文字の情報(文字列)をハミング符号で符号化し送信した。
あ 0000,い 0001,う 0010,え 0011,お 0100, か 0101,き 0110,く 0111, け 1000,こ 1001,さ 1010,し 1011,す 1100, せ 1101,そ 1110,た 1111
下記の受信した情報の誤りの有無をしらべ、誤りがあれば誤りのある箇所を特定し、送信さ れた情報(文字列)を復号せよ。

\[
    \begin{bmatrix}
        1011111 & 0111010 & 0001111 & 1000000 & 0100001 & 0001101 & 0100010 & 1010100
\end{bmatrix}
\]

\paragraph{ 回答 4\\}
各受信語 $w=(x_1,x_2,x_3,x_4,c_1,c_2,c_3)$ について,$e=H w^T$ を計算し,$e=0$で誤りなし,$e\neq0$で該当する列位置のビットを反転する。

1. $w_1=1011111$  
   $e=H w_1^T = [1,0,1]^T$ → $H$第2列

   第2ビット(x2)を反転:$1011111 \to 1111111$  
   データ部:1111 → 「た」

2. $w_2=0111010$  
   $w_2=(0,1,1,1,0,1,0)$  
   $e = H w_2^T = [1,1,0]^T$ → $H$第3列

   第3ビット(x3)を反転:$0111010 \to 0101010$  
   データ部:0101 → 「か」

3. $w_3=0001111$  
   $e=0$、誤りなし  

   データ部:0001 → 「い」

4. $w_4=1000000$  
   $e=[0,1,1]^T$ → $H$第1列

   第1ビット(x1)反転:$1000000 \to 0000000$  
   データ部:0000 → 「あ」

5. $w_5=0100001$  
   $e=[1,0,0]^T$ → $H$第5列

   第5ビット(c1)反転:$0100001 \to 0100101$  
   データ部:0100 → 「お」

6. $w_6=0001101$  
   $e=[0,1,0]^T$ → $H$第6列

   第6ビット(c2)反転:$0001101 \to 0001111$  
   データ部:0001 → 「い」

7. $w_7=0100010$  
   $e=[1,1,1]^T$ → $H$第4列

   第4ビット(x4)反転:$0100010 \to 0101010$  
   データ部:0101 → 「か」

8. $w_8=1010100$  
   $e=[0,0,1]^T$ → $H$第7列

   第7ビット(c3)反転:$1010100 \to 1010101$  
   データ部:1010 → 「さ」

よって,復号後の文字列は「たかいあおいかさ」となる。

\[
\boxed{\text{たかいあおいかさ}}
\]

\paragraph{ 問題 5\\}
生成多項式$G(x)=x^4+x^3+x^2+1$が与えられた時に、
情報(1 0 1)$=x^2 + 1 = P(x)$に対応する7ビットの巡回符号$F(x)$を求めよ。なお、解答は、2 進符号で記述すること。

※$F(x)=x^4P(x)+R(x)$、$R(x)$は$x^4P(x)$を$G(x)$で割った余り。

\paragraph{ 回答 5\\}
\[
\begin{aligned}
&\text{まず } x^4P(x)\text{ を求める:}\\
&x^4P(x)=x^4(x^2+1)=x^6+x^4 \\[6pt]
&\text{次に } x^6+x^4 \text{ を } G(x)=x^4+x^3+x^2+1 \text{ で割る。}\\[6pt]
1.\; &x^6+x^4 \div (x^4+x^3+x^2+1):
   \\
   &\text{商: }x^6 ÷ x^4 = x^2 \\
   &x^2 \cdot G(x) = x^6+x^5+x^4+x^2 \\[6pt]
   &\text{剰余: }(x^6+x^4) \oplus (x^6+x^5+x^4+x^2) = x^5+x^2 \\[6pt]
2.\; &x^5+x^2 \div (x^4+x^3+x^2+1):
   \\
   &\text{商: }x^5 ÷ x^4 = x \\
   &x \cdot G(x) = x^5+x^4+x^3+x \\[6pt]
   &\text{剰余: }(x^5+x^2) \oplus (x^5+x^4+x^3+x) = x^4+x^3+x+x^2 \\[6pt]
3.\; &(x^4+x^3+x^2+x) \div (x^4+x^3+x^2+1):
   \\
   &\text{商: } x^4 ÷ x^4 = 1 \\
   &1 \cdot G(x) = x^4+x^3+x^2+1 \\[6pt]
   &\text{剰余: }(x^4+x^3+x^2+x) \oplus (x^4+x^3+x^2+1) = x+1 \\[6pt]
&\text{よって余り }R(x)=x+1 \\[6pt]
&\text{よって符号語: } F(x)=x^4P(x)+R(x)= (x^6+x^4)+(x+1)=x^6+x^4+x+1 \\[6pt]
&F(x)=x^6+x^4+x+1 \text{ を2進列で表すと:}\\
&x^6:1,\; x^5:0,\; x^4:1,\; x^3:0,\; x^2:0,\; x^1:1,\; x^0:1 \\
&\boxed{1010011}
\end{aligned}
\]

\paragraph{ 問題 6\\}
生成多項式$G(x)=x^4+x^3+x^2+1$が与えられた時に、(1)~(3)の受信した符号語の剰余
(余り)を求め、誤りの有無を示せ。なお、誤りの根拠となる「剰余」と「誤りの有無」を明示すること。

(1) (1000101)($=x^6+x^2+1$)

(2) (1101001)($=x^6+x^5+x^3+1$)

(3) (0101110)($=x^5+x^3+x^2+x$)

(1) 受信語:$(1000101)_2 = x^6 + x^2 + 1$

\paragraph{ 回答 6\\}
\[
\begin{aligned}
x^6 + x^2 + 1 &\div (x^4 + x^3 + x^2 + 1) \\
&\xrightarrow{x^2 \text{で割る}} x^2 (x^4 + x^3 + x^2 + 1) = x^6 + x^5 + x^4 + x^2 \\
\text{剰余} &= (x^6 + x^2 + 1) \oplus (x^6 + x^5 + x^4 + x^2) = x^5 + x^4 + 1.
\end{aligned}
\]

続いて $x^5 + x^4 + 1 \div (x^4 + x^3 + x^2 + 1)$ を行う:
\[
\begin{aligned}
x^5 + x^4 + 1 &\xrightarrow{x \text{で割る}} x (x^4 + x^3 + x^2 + 1) = x^5 + x^4 + x^3 + x \\
\text{剰余} &= (x^5 + x^4 + 1) \oplus (x^5 + x^4 + x^3 + x) = x^3 + x + 1.
\end{aligned}
\]

これ以上次数が低い多項式($x^3 + x + 1$)なので割れず、剰余は
\[
R_1(x) = x^3 + x + 1 \neq 0.
\]
よって、(1)は誤り

(2) 受信語:$(1101001)_2 = x^6 + x^5 + x^3 + 1$

\[
\begin{aligned}
x^6 + x^5 + x^3 + 1 &\div (x^4 + x^3 + x^2 + 1) \\
&\xrightarrow{x^2 \text{で割る}} x^2(x^4 + x^3 + x^2 + 1) = x^6 + x^5 + x^4 + x^2 \\
\text{剰余} &= (x^6 + x^5 + x^3 + 1) \oplus (x^6 + x^5 + x^4 + x^2) = x^4 + x^3 + x^2 + 1.
\end{aligned}
\]

ここで、$x^4 + x^3 + x^2 + 1$ は $G(x)$ そのものなので、
\[
R_2(x) = 0.
\]
よって、(2)は誤りなし

(3) 受信語:$(0101110)_2 = x^5 + x^3 + x^2 + x$

\[
\begin{aligned}
x^5 + x^3 + x^2 + x &\div (x^4 + x^3 + x^2 + 1) \\
&\xrightarrow{x \text{で割る}} x (x^4 + x^3 + x^2 + 1) = x^5 + x^4 + x^3 + x \\
\text{剰余} &= (x^5 + x^3 + x^2 + x) \oplus (x^5 + x^4 + x^3 + x) = x^4 + x^2.
\end{aligned}
\]

さらに $x^4 + x^2 \div G(x)$ を行う:
\[
\begin{aligned}
x^4 + x^2 &\xrightarrow{1 \text{で割る}} (x^4 + x^3 + x^2 + 1) \\
\text{剰余} &= (x^4 + x^2) \oplus (x^4 + x^3 + x^2 + 1) = x^3 + 1.
\end{aligned}
\]

よって、
\[
R_3(x) = x^3 + 1 \neq 0.
\]
(3)は誤り

\[
\boxed{
(1)\ R_1(x)=x^3+x+1 \neq 0 \Rightarrow \text{誤り}\\
(2)\ R_2(x)=0 \Rightarrow \text{誤りなし}\\
(3)\ R_3(x)=x^3+1 \neq 0 \Rightarrow \text{誤り}
}
\]

\nocite{*}
\bibliographystyle{jplain}
\bibliography{refs}

\end{document}
